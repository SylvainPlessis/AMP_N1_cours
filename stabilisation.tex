\begin{frame}{La stabilisation}
{Eurêka!\only<4->{ $\Rightarrow$ Applications}}
\only<-3>{%
\visible<3>{\centering La flottabilité est le bilan de ces deux tendances.}
\juxt{%
\only<2>{Plus on est lourd, plus on coule.}%
\includegraphics<1>[width=\textwidth]{flottabilite}%
\includegraphics<3>[width=\textwidth]{flottabilite3}%
}{%
\only<1>{Plus on est volumineux, plus on flotte}%
\includegraphics<2>[width=\textwidth]{flottabilite2}%
\includegraphics<3>[width=\textwidth]{flottabilite4}%
}}%
\only<4-7>{%
\juxt{%
\begin{itemize}[<+(3)->]
\item La combinaison ajoute du volume
\item[$\Rightarrow$] il faut des plombs pour la compenser.
\end{itemize}%
}{%
\begin{itemize}[<+(3)->]
\item À la descente, la combinaison s'écrase sous l'effet de
  la pression
\item[$\Rightarrow$] il faut gonfler son gilet pour compenser.
\end{itemize}%
}}%
\only<8->{%
\juxt[0.3]{%
\includegraphics[width=0.8\textwidth]{test_flotta}
}{%
\begin{exampleblock}{Test de lestage}
En surface, poumons et gilet vides, l'eau est au niveau
du masque.\\
\alert{On ne coule pas!}\\
\uncover<9>{Se confirme et s'évalue en fin de plongée:\\%
$\left.\begin{array}{l}
\text{50~bars}    \\
\text{3~m}        \\
\text{gilet vide} \\
\end{array}\right\}\Rightarrow$ flottabilité neutre}
\end{exampleblock}
}}
\end{frame}
